\section{Introduction}
\label{sec:introduction}

Robots have been used in many industrial applications, such as pick and place, spray-painting, and spot-welding. In those applications, the robots either do not need very high accuracy (such as in material handling for large objects), or the robots are programmed online (programming by teaching), where the important parameter is the repeatability of the robot instead of the accuracy. Repeatability refers to the capability of the robot to return to the same location as previously taught precisely, whereas accuracy refers to the capability of the robot to achieve a location (computed based on the robot kinematics model) in the robot workspace. If the robots are taught by teaching pendants to reach certain locations, the accuracy of the robot becomes irrelevant to the task. 

However, there are a lot of interests to use robot on many other applications where the accuracy of the robot becomes very crucial, in which the robot has to adapt to each task automatically with a great precision. Take, for example, a robot drilling task. In robot drilling task, the robot is supposed to drill several holes at precisely-defined locations on a workpiece, which can change for each task. We can teach the robot the position of the holes manually for each workpiece, but that will take a lot of time and effort. If we want to program the robot offline for such a task, the robot has to be able to do a few things: the robot has to scan the workpiece, determine the location for the holes, and finally
move to that location accurately. The accuracy of such a robotic system depends on several factors: The accuracy of the robot arm, the accuracy of the measurement system, and the accuracy of the end effector tool. 

The accuracy of the robot arm is determined by how closely the kinematic parameters of the robot model resembles the actual kinematic parameter of the physical robot. This is affected by the manufacturing process, the assembly process, and the wear and tear during the operation of the robot. To achieve a better accuracy, a robot calibration is normally conducted, either by using an external measurement system (such as motion capture system, or spinarm, or CMM), or by constraining the motion of the end effector. 

The accuracy of the measurement system can be divided into two parts: the accuracy of the measurement device itself, and the accuracy of the homogeneous transformation between the measurement frame to the robot frame (often called the extrinsic parameters). A camera system, for example, is generallly less precise as compared to a laser system, although a camera can give more information. To obtain the accurate extrinsic parameters of a measurement device, the information from the CAD model is not sufficient, as there will be error during the assembly, and very often the measurement coordinate is located inside the measurement device, causing it to be difficult to be measured accurately. When the measurement device is attached to the robot, a process known as hand-eye calibration is normally used to find the extrinsic parameters. 

Lastly, the accuracy of the end effector transformation refers to the homogeneous transformation from the robot flange coordinate frame to a pre-determined location on the end effector (such as the tip of the drilling bit). Calibration of the end effector is generally known as TCP calibration. 

In this paper we concern ourselves with solving the first two issues. We propose to use a near-range 2D \ac{lrf}, attached on a robot arm, to calibrate the kinematics parameters of the robot and at the same time also calibrate the extrinsic parameters of the \ac{lrf}. The \ac{lrf} is chosen because it can give very accurate measurement data, both for the calibration and for the subsequent task (such as drilling). The method does not require any other external measurement setup, except three approximately perpendicular planes located around the robot.

The overall procedure is as follows:
\begin{enumerate}
\item The scanner is attached on the robot, and three (approximately) perpendicular planes are set around the robot. The location (position and orientation) of the planes only needs to be known approximately.
\item The robot is moved to several positions, such that the \ac{lrf}'s 2D ray intersects each of the plane (one at a time). The data from the \ac{lrf} as well as the joint angles information are collected.
\item An optimization algorithm (Levenberg-Marqueadt) is used to find the optimal robot kinematics parameters, together with the \ac{lrf} extrinsic parameters, by satifsfing the following constraints: the projected \ac{lrf} data should fall on the three planes. 
\end{enumerate}

The remainder of the paper is as follows. In \sref{sec:related}, we discuss the existing approaches to the calibration problem, both for the robot kinematics and the \ac{lrf} extrinsic parameters, and how our proposed method differs from the others. In \sref{sec:method}, the proposed method is explained in detail. A simulation study is presented in \sref{sec:simulation} to verify the method, and finally we conclude the paper with a few remarks in \sref{sec:conclusions}.  

\section{Related works}
\label{sec:related}
\subsection{Calibration of robot kinematic parameters}
\label{sec:kine_calib}
Robot kinematics calibration have been researched for quite a long time; some of the earliest work began in 1980s. Generally, the calibration process can be divided into unconstrained and constrained calibration. In unconstrained calibration, the robot moves its end-effectors to several poses freely, while an external measurement system measures the pose. The measured pose is then compared to the one computed from the kinematics model, and the model can be updated to minimized the difference between the model and the actual pose. 

For example, Ginani and Mota \cite{Ginani2011} calibrate an ABB IRB 2000 industrial robot using a ROMER measurement arm, which improves the mean/maximum position errors from 1.25mm/2.20mm to 0.30mm/1.40mm. Ye et al. \cite{Ye2006} used a Faro laser tracker to calibrate an ABB IRB 2400/L, and improve the mean/maximum position errors from 0.963mm/1.764mm to 0.470 mm/0.640mm. In \cite{Nubiola2013}, Nubiola and Bonev used a Faro laser tracker ION to calibrate an ABB IRB 1600-6/1.45 robot. The mean/maximum position errors are reduced from 0.968mm/2.158mm to 0.364 mm/0.696 mm, respectively.

The issue with such calibration process is the expensive cost of the external measurement system. For example, the cost of a laser tracker is more than $100,000$\cite{Nubiola2013}. Therefore, many researchers try to find calibration methods which only rely on the internal sensors of the robot, and by constraining the motion of the end-effector to provide the calibration equations. 

In \cite{Ikits1997}, Ikits and Hollerbach propose a kinematic calibration method using a planar constraint. The robot end effector (a touch probe) is moved to touch random points on a plane. When the touch probe is in contact with the plane, the joint angles are recorded. The kinematics parameters of the robot model are then updated to satisfy the planar constraint. While the approach is promising, they also report observability problems for some of the calibration parameters.  

In \cite{Zhuang1999}, Zhuang et al. investigated robot calibration with planar constraints, in particular the observability conditions for the parameters of the robot kinematic model. They proved that a single-plane constraint is insufficient for calibrating a robot,and a minimum of three-plane constraint is necessary. Using a three-plane constraint, the constraint system is proved to be equivalent to an unconstrained point-measurement system under three conditions: a) All three planes are mutually non-parallel, b) The identification Jacobian of the unconstrained system is nonsingular, and c) Measured points from each individual plane do not lie on a line on that plane. They verified the theory by doing a simulation on a PUMA560 robot. 
In \cite{Joubair2015}, Joubair and Bonev calibrated both the kinematic and non-kinematic (stiffness) parameters of a FANUC LR Mate 200iC industrial robot by using planar constraints, in the form of a high precision 9-in. granite cube. The robot is equipped with an MP250 Renishaw touch probe, which is then moved to touch four planes of the granite cube. The granite cube's face is flat to within 0.002mm. They improved the maximum plane error from 3.740mm to 0.083mm. 

\subsection{Calibration of extrinsic 2D \ac{lrf} parameters}
\label{sec:laser_calib}
Extrinsic calibration of an \ac{lrf} consists of finding the correct homogeneous transformation from the laser coordinate frame to the robot coordinate frame. While it is similar to hand-eye calibration of a robot-camera system, the data obtained from an \ac{lrf} is less informative than a camera, which makes it slightly more difficult. A camera on a robot is normally calibrated by using a checkerboard pattern, the pose of which (position and orientation) with respect to the camera can be obtained by the camera using just a single image. In contrast, we cannot obtain the pose of an object easily using just a single data from an \ac{lrf}. 

Most of the works on extrinsic calibration of \ac{lrf} involves a camera, since both sensors are often used together in many applications. The works are mostly based on Zhang and Pless work\cite{Zhang2004}. They proposed a method to calibrate both a camera and an \ac{lrf} using a planar checkerboard pattern. First, the camera is calibrated by using a checkerboard pattern, using a standard hand-eye calibration. The calibrated camera is then used to calculate the pose of the pattern. Next, the robot is moved to several poses with the \ac{lrf} pointing to the pattern. By using the geometric constraints that all the data points from the \ac{lrf} should fall on the pattern plane, the extrinsic parameters of the \ac{lrf} can be obtained. Finally, the same constraints is used to optimize both the intrinsic and extrinsic parameter of the camera and the extrinsic parameter of the \ac{lrf}. The nonlinear optimization problem is solved with the Levenberg-Marquardt method.

Unnikrishnan and Hebert \cite{Unnikrishnan2005} used the same setup as \cite{Zhang2004}, although they do not optimize the camera parameter simultaneously due to the nonlinearity of the resulting cost function. 
Li et al. \cite{Li2007} used a specially designed checkerboard to calibrate the extrinsic parameters between a camera and an \ac{lrf}, and claim that the result is better than \cite{Zhang2004}. Vasconcelos et al. \cite{Vasconcelos2012} developed a minimal closed-form solution for the extrinsic calibration of a camera and an \ac{lrf}, based on the work in \cite{Zhang2004}. 

Our proposed algorithm can be seen as a combination of the algorithm for extrinsic calibration of \ac{lrf} \cite{Zhang2004} and the algorithm for calibration of the robot kinematics parameter using three planar constraints \cite{Joubair2015}. The proposed method simultaneously optimize both the \ac{lrf} extrinsic parameters and the robot kinematics parameters to satisfy the planar constraints. It has the following advantages as compared to the other calibration approaches:
\begin{enumerate}
\item The method does not need an additional camera to calibrate the \ac{lrf}, unlike \cite{Zhang2004}
\item The method does not need another expensive external measurement system. The measurement is done using the \ac{lrf}, which will also be used in the robot task, so it does not incur additional cost. Moveover, \ac{lrf} can achieve very high accuracy at much lower cost, as compared to measurement system such as Vicon or Faro Laser Tracker. 
\item The method does not need a precisely manufactured calibration object such as the granite cube in\cite{Joubair2015}, whose planes location and orientation are supposed to be known accurately. The method only requires
three flat surfaces, which are oriented somewhat perpendicularly, and the location and orientation are known only approximately. 
\item The calibration setup can be done very easily, since the planes do not need to assume precisely known locations.  
\item The calibration poses can be distributed throughout the whole workspace, instead of being confined only in a local region like in \cite{Joubair2015}. 
\end{enumerate}



%%% Local Variables:
%%% mode: latex
%%% TeX-master: "../main"
%%% End:

\section{Simulation}
\label{sec:simulation}

The proposed method is verified through simulations of the calibration procedure. The simulation is conducted by using ROS and Gazebo, where the robot model, the \ac{lrf}, and the planes can be simulated.  
As shown in \fref{fig:robot_setup}, three perpendicular planes are located around the robot, and the robot is moved such that the \ac{lrf} ray intersects each of the plane. Simulated data from the \ac{lrf} can be obtained, and Gaussian noise can be added. The data is then used as input to the calibration procedure. 

The robot kinematics model in the simulation is considered as having the "true" kinematics parameters, and another kinematics model is generated by introducing random gaussian errors to the true parameters within the range of $\pm$ 2mm and $\pm$ 1$^o$ for the linear and angular parameters, respectively. Note that the errors are considered to be very big, and they are purposedly set to be large to illustrate the robustness of the calibration method. 

After the calibration procedure, the robot is moved to 1000 random poses to evaluate the proposed calibration method. Let $\vb*{{^B}T_{L,j}^{true}}$ and $\vb*{{^B}T_{L,j}^{model}}$ be the true and calibrated pose of the \ac{lrf} frame w.r.t. the robot base at the robot pose index j, respectively, then the error can be computed as follows. 
\begin{equation}
\Delta T_j =  (\vb*{{^B}T_{L,j}^{model}})^{-1} \vb*{{^B}T_{L,j}^{true}}
\end{equation}
Let $\delta_{uv}$ be the element of $\Delta T_j$, then the error in position, $\delta p_j$, can be computed by
\begin{equation}
\delta p_j = \sqrt{\delta_{14}^2 + \delta_{24}^2 + \delta_{34}^2}
\end{equation}.
Let $\delta R_j$ be the rotation part of the homogeneous transformation matrix $\Delta T_j$. $\delta R_j$ can be represented by using an axis-angle notation, $[r_{j,1}\; r_{j,2}\; r_{j,3}\; \delta \theta_j]$. $\delta\theta_j$ is the error in rotation (it can be seen as the amount of rotation necessary to turn the calibrated pose to the true pose). 
By using the proposed error formula, the initial average error (due to the random errors introduced to the model) is 21mm and 4$^o$.



\subsection{The effect of measurement accuracy}
\label{sec:meas_accuracy}
The accuracy of the calibration procedure greatly depends on the accuracy of the measurement system. In this section, the calibration is conducted at different level of \ac{lrf} accuracy, as shown in \fref{fig:laser_accuracy}. The accuracy is adjusted in the simulation by changing the gaussian noise. As we can see, the errors decrease as the accuracy of the \ac{lrf} improves. At 0.1mm accuracy (which can be achieved by an \ac{lrf} very easily), the position error and the orientation errors are around 0.1mm and 0.02$^o$, respectively. 
\begin{figure}[h]
  \centering
  \subfloat{\includegraphics[height=40mm]{laser_pos}}\;
  \subfloat{\includegraphics[height=40mm]{laser_ori}}
  \caption{Effect of the laser accuracy towards: a) Linear Accuracy b) Angular Accuracy} 
  \label{fig:laser_accuracy}
\end{figure}


\subsection{The effect of the number of calibration poses}
\label{sec:calib_poses}
In \fref{fig:num_of_poses}, it can be seen that as the number of poses increases, the accuracy improves until around N=20, beyond which it does not change much. It can be concluded that 20 robot poses are sufficient to calibrate the robot model accurately. Note that N refers to the robot pose w.r.t. to one plane, so the total poses will be 3N = 60 poses. 

\begin{figure}[h]
  \centering
  \subfloat{\includegraphics[height=40mm]{num_of_poses_pos}}\;
  \subfloat{\includegraphics[height=40mm]{num_of_poses_ori}}
  \caption{Effect of the number of poses towards: a) Linear Accuracy b) Angular Accuracy} 
  \label{fig:num_of_poses}
\end{figure}

In addition, we also vary the number of points per pose from the \ac{lrf} data that are used for calibration \ref{fig:num_of_points}. Note that after around 10 points, the calibration result does not improve significantly. Therefore, although an \ac{lrf} can give more than 300 data points per pose, using more than 10-20 points do not improve the calibration by much. 

\begin{figure}[h]
  \centering
  \subfloat{\includegraphics[height=40mm]{num_of_points_pos}}\;
  \subfloat{\includegraphics[height=40mm]{num_of_points_ori}}
  \caption{Effect of the number of points towards: a) Linear Accuracy b) Angular Accuracy} 
  \label{fig:num_of_points}
\end{figure}


\subsection{The effect of the plane parameters' initial estimate}
\label{sec:plane_params}
One of the advantage of the proposed method is that the planes parameters do not need to be precisely known. Here we vary the planes parameters estimate, to demonstrate the robustness of our method. The estimated normals and position of the planes are disturbed by as much as 30$^o$ and 65mm, as shown in \fref{fig:plane_params}. It is obvious that the errors in the planes' parameters estimate do not affect the calibration accuracy at all. In fact, after calibration, the planes' parameters in the calibrated model approach the true parameters very closely (within 0.1mm and 0.02$^o$.


\begin{figure}[h]
  \centering
  \subfloat{\includegraphics[height=40mm]{plane_param_pos}}\;
  \subfloat{\includegraphics[height=40mm]{plane_param_ori}}
  \caption{Effect of the accuracy of the plane parameters initial estimate towards: a) Linear Accuracy b) Angular Accuracy} 
  \label{fig:plane_params}
\end{figure}




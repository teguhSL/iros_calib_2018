\section{Simulation}
\label{sec:simulation}

We verify SCALAR through simulation of the calibration procedure. The simulation is conducted by using Robot Operating System and Gazebo where the robot model, the \ac{lrf}, and the three planes can be simulated.  
As shown in \fref{fig:robot_setup}, three perpendicular planes are located around the robot, and the robot is moved such that the \ac{lrf}'s ray intersects each plane. Simulated data from the \ac{lrf} can be obtained and Gaussian noise with zero mean and standar deviation $\sigma_{\rm{noise}}$ can be added to the data. The data is then used as input to the calibration algorithm. The optimization is typically completed in less than 15 s using Python.

After the calibration, the robot is moved to 10,000 random poses to evaluate the accuracy of the calibrated parameters. Let ${^B}\vb*{T}_{L,j,true}$ and ${^B}\vb*{T}_{L,j,model}$ be the true and calibrated pose of the \ac{lrf}'s frame w.r.t. the robot base frame at the robot pose index $j$, respectively, then the error of the calibrated model can be computed as 
\begin{equation}
\Delta \vb*{T}_j =  {{^B}\vb*{T}_{L,j,model}}^{-1} \; {^B}\vb*{T}_{L,j,true} \; .
\end{equation}
Let $\delta t_{uv}$ be the element of $\Delta \vb*{T}_j$ with the subscript $u$ and $v$ refer to the row and column index, then \textbf{the position error} at the robot pose index $j$, $\delta p_j$, can be computed by
\begin{equation}
\delta p_j = \sqrt{\delta t_{14}^2 + \delta t_{24}^2 + \delta t_{34}^2} \quad .
\end{equation}
Let $\delta \vb*{R}_j$ be the rotation part of the homogeneous transformation matrix $\Delta \vb*{T}_j$. $\delta \vb*{R}_j$ can be represented by using an axis-angle notation, $[r_{j,1}\quad r_{j,2}\quad r_{j,3}\quad \delta \theta_j]$. We use $\delta\theta_j$ as \textbf{the orientation error} at the robot pose index $j$.  $\delta\theta_j$ can be seen as the amount of rotation necessary to rotate the calibrated pose to the true pose. The errors $\delta p_j$ and $\delta\theta_j$ are then averaged over the 10,000 random poses.


\begin{figure*}[t]
  \centering
  \subfloat[]{
  \includegraphics[height=38mm]{laser_pos}\;
  \includegraphics[height=38mm]{laser_ori}
  \label{fig:laser_noise}
  }\quad
  \subfloat[]{
  \includegraphics[height=38mm]{num_of_poses_pos}
  \includegraphics[height=38mm]{num_of_poses_ori}
  \label{fig:num_of_poses}
  }\\[-0.4ex]
  \centering
  \subfloat[]{
  \includegraphics[height=39mm]{plane_param_linear_pos}\;
  \includegraphics[height=39mm]{plane_param_linear_ori}
  \label{fig:plane_params_linear}
  }\quad\quad
  \subfloat[]{
  \includegraphics[height=39mm]{plane_param_angular_pos}\quad
  \includegraphics[height=39mm]{plane_param_angular_ori}
  \label{fig:plane_params_angular}
  }\quad  
   \caption{Effect of a) the measurement noise, b) the number of poses, c) planes' position estimate error, and d) plane's orientation estimate error towards the position and orientation error after calibration}
\end{figure*}

The simulated robot's model is considered as having the true kinematics and extrinsic parameters, and an initial model is generated by introducing random Gaussian errors to the true parameters within the range of $\pm$ 2 mm and $\pm$ 1$^o$ for the linear and angular parameters, respectively. Note that the initial model's errors are intentionally set to be large to illustrate the robustness of the calibration method. The average position and orientation errors of the initial model as compared to the true model are 14.6 mm and 4.05$^o$, while the maximum errors are 103.9 mm and 5.95$^o$. We run the calibration procedure to improve the initial model with $3N$ = 120, $M$ = 100 and $\sigma_{\rm{noise}}$ = 0.1mm, and the resulting calibrated model has the average position and orientation errors reduced to 0.09 mm and 0.02$^o$, while the maximum errors are reduced to 0.19 mm and 0.035$^o$.

Next, we evaluate the effect of the measurement noise, the number of poses ($N$) and points ($M$), and the plane parameters' initial estimate on the calibration errors. 


\subsection{Effect of the measurement noise}
\label{sec:meas_accuracy}
The accuracy of the calibration procedure depends greatly on the accuracy of the measurement system, which is affected by the noise on the data. In this section, Gaussian noise with zero means and varying standard deviations $\sigma_{\rm{noise}}$ are added to the measurement data in the simulation, and its effect on the calibration errors is shown in \fref{fig:laser_noise}. As $\sigma_{\rm{noise}}$ decreases, the calibration errors decrease. At $\sigma_{\rm{noise}}$ = 0.1 mm, the average position and orientation errors are around 0.1 mm and 0.02$^o$, respectively. For the subsequent sections, $\sigma_{\rm{noise}}$ is set at 0.1 mm. 

\subsection{Effect of the number of calibration poses and the number of points}
\label{sec:calib_poses}
For each plane, the robot is moved to $N$ poses, so in total there are $3N$ calibration poses. At each pose, we select $M$ data points from the \ac{lrf} data. In this section we evaluate the effect of $3N$ and $M$ to the calibration errors. In \fref{fig:num_of_poses}, it can be seen that as the number of poses $3N$ increases, the error decreases until around $3N$ = 60, beyond which it does not change significantly. It can be concluded that 60 robot poses are sufficient to calibrate the robot model accurately. We also conduct similar analysis on $M$ (not presented in this paper) and found that $M$ = 20 is sufficient for the calibration.


\subsection{Effect of the plane parameters' initial estimate}
\label{sec:plane_params}
One of the advantages of SCALAR is that the plane parameters do not need to be precisely known. Here the plane parameters' estimate is varied to demonstrate the method's robustness . The initial estimates of the positions and the normals of the planes are disturbed by up to 100 mm and 30$^o$, as shown in \fref{fig:plane_params_linear} and \fref{fig:plane_params_angular}. From the figures, it can be seen that the calibration position and orientation errors are not affected by the errors in the plane parameters' initial estimate. In fact, after calibration, the plane parameters in the calibrated model approach the true parameters within 0.1 mm and 0.01$^o$.






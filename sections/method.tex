\section{Method}
\label{sec:method}

The calibration setup is depicted in figure ***, where three perpendicular planes are installed around the robot. A laser range finder is attached on the robot flange. For each plane, the robot is moved to $N$ poses such that the laser line falls on the respective plane. The joint angles of the robot and the data from the LRF are then recorded for the calibration. 

This section provides the detail on how to calibrate both the extrinsic parameters of the LRF and the robot kinematics parameters. First, the initial estimate of the LRF extrinsic parameter is obtained by using linear least-square method. After that, the LRF extrinsic parameters and the robot kinematics parameters are optimized by using nonlinear optimization method. 
Finally, SVD is used to identify which parameters are identifiable, and the steps to handle the unidentifiable parameters are then presented. 
\subsection{Obtaining Initial Estimate of the LRF Extrinsic Parameters}
\label{sec:first_step}









\subsection{Optimizing LRF Extrinsic Parameters and Robot Kinematic Parameters}
\label{sec:second_step}





\subsection{The identifiability of the optimization parameters}
\label{sec:third_step}




\subsection{Handling the unidentifiable parameters}
\label{sec:fourth_step}
